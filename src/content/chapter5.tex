\chapter{A Roadmap for Future Work}
\label{chapter:future_work}

Provenance Replay as currently implemented provides strong initial support for
in silico reproducibility in QIIME 2, but much additional work will be required
in order to fully achieve the goals set out in the introduction. Chief among the
concerns are:

\begin{itemize}
    \item the development of additional components of a comprehensive reproducibility package
    \item the reproducibility of QIIME 2 computing environments
    \item the creation of additional interfaces and target interfaces
\end{itemize}

We will explore these and lesser goals at a high level here, falling back on
technical communication tools like the \texttt{provenance\_py} issue tracker
\parencite{keefe_issues_2021} for low-level commentary and technical detail.
Some of these ideas are fully fledged, but many are aspirational. Where open
issues exist for an item, they will be referenced by number in the item title to
allow readers to confirm completion status.

\section{A comprehensive reproducibility package}

As discussed above, in silico reproducibility requires the reproducer to be in
possession of the original, or similar study data and sample metadata, computing
resources, and complete reporting on the analytical methods used. To make the
practice of documenting provenance truly “too easy not to do”, we must attempt
to cover all of these requirements, and we must make it simple. Provenance
Replay provides a basic “one-shot” tool for reporting study provenance -
augmenting it with additional reports will improve the completeness of
provenance documentation while reducing the time-cost required to do so.

\subsection{A methods manifest \parencite[Issue 75]{keefe_issues_2021}}
\label{methods_manifest}

Because provenance replay is organized in terms of QIIME 2 Actions, the
production of an ordered manifest of computational methods is possible.
Anecdotally, methods-section writing in computational biology papers requires
authors to find a balance between completeness and approachability. Few readers
are likely to care about the details of every method you have used in a
large-scale project, but a reader attempting to reproduce your study might
benefit significantly from a plaintext description of the steps you have taken.

Brief descriptions of QIIME 2 actions, registered via the plugin registration
API, can be written to disk, alongside the name of the plugin and action that
were used, and reference keys that map to the keys in an output bibliography.
Depending on the complexity of the tooling required, these could be simple
numerical keys managed in Python, or this report and the reference list itself
could be produced with LaTeX/BibTeX.

Replay scripts, the methods manifest, and the methods reference list could be
submitted for publication as an appendix or supplement, allowing authors to
defer comprehensive methods descriptions in favor of references to the scripts
and manifests, while still providing complete attribution to the authors of the
methods applied during the analysis. When replay is conducted on all analysis
results (rather than a curated set), readers will have better transparency into
which Results were not reported.

\subsection{Analysis metadata \parencite[Issue 76]{keefe_issues_2021}}

There is a class of information that, while not strictly required for
reproducibility, has the potential to make provenance replay much more pleasant
to use. A brief report of this information, where available, could be
well-received. A summary of data Artifact UUIDs and/or file path names could
encourage good data sharing practices by helping users identify the files they
need to share. Details on the size and type of this data could help with
estimating compute times. Overall and per-action summaries of runtime and
hardware provisioning could simplify the process of re-running jobs.

\subsection{Improving computer system reporting \parencite[Issue 77]{keefe_issues_2021}}
\label{comp_env_reporting}

Computing environment reproducibility is currently served by a basic report  on
the operating systems, versions of the QIIME 2 framework and plugins used in an
analysis. With some effort, one or more computing environments could be
assembled from these with a high probability of successfully replaying an
analysis. Improving the sophistication of this reporting may be a useful
short-term goal. Associating these computing environment data with specific
methods (e.g. through the methods manifest) would make manual cross-version
reproducibility much more tractable. Automating the creation of computing
environments is the better long-term target, and will be discussed in Section
\ref{automating_comp_env_repr}.

Many underlying dependencies and system features are not captured in provenance,
negatively impacting the completeness of our reporting: hardware and
architecture characteristics, hardware allocations (e.g. in cluster
environments), and non-Python dependencies, for example. Capturing some of these
from within the Python Framework itself may be intractable, but interfaces
designed to e.g. interact with cluster scheduling tools (\texttt{q2slurm},
anyone?) could make job parameter capture possible, allowing for the inclusion
of this data in provenance. This feature was frequently requested by focus group
participants, and could significantly improve learning and automation goals.
Generally, improving capture and reporting of the computer system would advance
completeness goals, and open the door to many positive outcomes.


\section{Automating Computing Environment Reproducibility \parencite[Issue 78]{keefe_issues_2021}}
\label{automating_comp_env_repr}

Computing environment reproducibility is frequently one of the more challenging
parts of in silico reproducibility work, and Provenance Replay does not support
it adequately at this time. The QIIME 2 distribution model is in the middle of a
transition from a “core distribution” model, where QIIME 2 is packaged with a
“core” plugin set, to a model in which plugins or sets of plugins can be
selected at will from the QIIME 2 library. QIIME 2 users benefit from the
ability to run analyses across multiple Framework and plugin versions, but these
different versions frequently require conflicting dependencies, and installing
them side by side is not always recommended.

QIIME 2 currently uses conda environments, docker images, and virtual machines
to safely manage these conflicting dependencies through sequestration.
Re-running an analysis across multiple environments is not currently possible
without significant user intervention - breaking the output script into
environment-specific parts, activating and deactivating environments, and
running those parts. Undertaking this process manually (with the reporting
described in Section \ref{comp_env_reporting}) would be unpleasant but feasible.

A better approach would involve tooling for the aggregation of software
dependencies, automatic generation of conda or similar environments, and the
output of a replay script capable of transitioning between these environments as
required. The bones of this latter can be found in the methods manifest
described in Section \ref{methods_manifest}. Full support for this type of
replay could be expensive to develop, and would primarily be of value in
simplifying the corroboration of results, and in the automation of large-scale
analyses. 

Provenance Replay currently assumes that replay will occur in the QIIME 2
environment used to generate the replay scripts. Support for multi-environment
replay would allow us to fully trust that the plugin, action, and parameter
names captured in provenance would match those expected by the active
environment, without requiring input from the Plugin Manager.

\section{Diversity in interfaces and usage drivers (\cite[Issue 65]{keefe_issues_2021}; \cite[Issue 79]{keefe_issues_2021})}

Readability, learning-readiness, and communication/collaboration goals all
benefit from an increased number and diversity of provenance replay interfaces,
and usage drivers targeting different QIIME 2 interfaces for replay. Focus group
participants described analysis workflows centered on jupyter notebooks in both
the QIIME 2 Python 3 API and the command line interface (in a notebook running
the BASH kernel). Usage drivers targeting the \texttt{.ipynb} format are likely to be
relatively straightforward, and will greatly improve users ability to learn from
and perform analyses interactively.

GUI support for provenance replay through the q2Galaxy interface would be useful
as q2galaxy gains traction. Replay “script” generation targeting the same
interface could inherit from the GalaxyDriver, templating out a series of
actions the user should perform within the GUI in order to replay an analysis.

In the mid to long term, R interface support would help support valuable
cross-language analysis “translation” outcomes.

\section{Improved in-memory replay support \parencite[Issue 22]{keefe_issues_2021}}
QIIME 2’s Python API provides strong support for interactivity, and can speed up
analysis noticeably in some cases by removing the requirement that Results in
memory be saved to disk. Provenance Replay does not currently provide support
for the replay of Results in memory, reducing the utility of its Python API
significantly. As a stopgap measure, replay of these in-memory Archives will be
implemented by saving Results to a temporary directory and then parsing them, at
significant I/O cost. This will provide a stable API for users, but no immediate
performance benefits.

Proper support for in-memory Results can be implemented by teaching the
\texttt{\_ArchiveParser} how to interact with non-ZIP Archives in memory. Because the
archive structure will be identical to that of a compressed \texttt{.qza} or \texttt{.qzv} on
disk, the parsing logic can remain the same. In order to support this, the
\texttt{\_ArchiveParser} module will need to be refactored to extract an internal API
which can be implemented for specific Archive variants and injected into the
\texttt{Parser}. Looking forward, support for parsing artifacts from the Artifact Cache
system (pending development) may provide a more robust approach to in-memory
results parsing.

\section{Simplifying the corroboration of results}

\subsection{“Touchless replay” \parencite[Issue 63]{keefe_issues_2021}}
\label{section:touchless_replay}

Claerbout’s “automatic” replay of figures is an inspiring example of how easy
basic methods reproduction should be. This type of “touchless” replay, in which
a Result or \texttt{ProvDAG} is replayed automatically into the same Result or \texttt{ProvDAG},
is blocked in the context of QIIME 2 by two key issues.

First, strict corroboration of results produced through stochastic processes is
at least as challenging in QIIME 2 as it is elsewhere. Stochastic processes that
could be controlled by setting a seed value do not always do so in a way that
allows provenance to capture that seed value. \texttt{q2\_feature\_table}’s \texttt{rarefy},
for example, is used heavily for random subsampling in QIIME 2, but does not allow
users to pass a seed as a parameter, rendering strict corroboration of output
values impossible.

Second, downstream Results do not contain the raw data required for their own
replay. “Touchless” replay in QIIME 2 would, at a minimum, require some user to
provide an input data file for importing. A touchless replay that accepts these
data from the user could be constructed, but would be limited to strict
corroboration, and would be a more complex problem than it initially appears.
Many analyses begin with the import of multiple pieces of raw data, and a
user-friendly approach for mapping the correct data inputs to the correct import
commands would likely require both interactivity and some user expertise if they
targeted human users.

A more viable long-term solution would be to develop tooling for the QIIME 2
platform capable of managing (or at least tracking) relevant data on disk. This
could be built around a database mapping QIIME 2 data inputs, metadata, and
Results to paths in some managed filesystem (or area of a file system), and
would allow a storage-aware usage-driver to conduct touchless provenance replay
reliably.

Alongside these challenges/costs of development, the limited value of strict
corroboration (as compared to the potential value of study extension to new data
or new methods) to the average human user of provenance replay has made it a
low-priority target in the short term. Automation-oriented user stories like
those of the QIITA meta-analysis targeting optimization from Section \ref{potential_users}
make this an attractive mid to long-term target for development.

\subsection{Variables in scripts and guaranteeing metadata identity \parencite[Issue 44]{keefe_issues_2021}}

In addition to input data (e.g. sequences), the production of almost all Results
requires one or more sample metadata tables/columns. One Focus Group participant
suggested that replay scripts with all user-sourced data inputs assigned to
variables at the top of the script might be easier to use. Though I doubt this
to be true for e.g. sequence data inputs, it would certainly be true for sample
metadata, which tends to be reused frequently during many analyses.

Assigning Metadata inputs to variables at the top of replay scripts would save
users some toil finding/replacing the same value repeatedly, and should be a
target for future development. Unfortunately, QIIME 2’s provenance capture
system does not capture enough information about sample metadata for us to
understand its identity and uniqueness (or lack thereof).

Metadata is captured in tabular form in QIIME 2, with each Action’s metadata
stored as a \texttt{.tsv} in the Action’s UUID-named provenance directory. These \texttt{.tsvs},
once extracted, can be referenced in a replay script, allowing for replay to
target strict corroboration (original data, original metadata). In this
scenario, each replay Action will receive a separate file path to the separate
\texttt{.tsv} originally stored in its provenance, even if only one metadata sheet was
used for all of the Actions in the original analysis. Until QIIME 2 captures
enough information to confirm the identity of Metadata, replay cannot know how
many Metadata inputs are required, or which Action uses which Metadata.

Pull request \#464 on the \texttt{qiime2} repository \parencite{qiime_2_development_team_pull_2016}
provides more sophisticated Metadata tracking in Provenance, and should be
considered reviewed for adequacy in meeting this requirement before development
effort is put into these goals.

\section{Improving performance for large-scale analyses \parencite[Issue 29]{keefe_issues_2021}}

Over 95\% of the runtime of any replay action on a large (many-Results) analysis
can be attributed to parsing, and adequate support for large-scale automation
workflows will likely require improvements in performance. The reading of
provenance files is an I/O-intensive operation, and though formal profiling has
not been conducted, it is likely to be the source of the bottleneck. Efforts to
reduce the number of reads are likely to improve runtimes. Provenance parsing
currently skips any Results whose root UUID is already contained in the \texttt{DiGraph}
being constructed, as all tracked predecessors of a node must also be in the
\texttt{DiGraph} when this occurs. A more granular approach can reduce redundant reads
dramatically, if the parser skips any nodes within a Result that already exist
in the \texttt{DiGraph}.

\section{Graph operations for meta-analysis \parencite[Issue 30]{keefe_issues_2021}}
\label{section:graph_ops}

The \texttt{DiGraph} structure of provenance lends itself to a number of graph operations
with potential to provide value to users. Graph intersection allows users to ask
whether two analyses contain any common components, and symmetric difference
allows users to ask what the unique characteristics of two analyses are. These
could be applied to UUID-based \texttt{DiGraph}s. In an academic context, the ability to
“diff” analyses by UUID would allow instructors to identify some violations of
academic integrity with student Results alone. More interesting questions could
be asked about the ways in which QIIME 2 analyses are performed by relabeling
the graph nodes with meaningful data - action names, for example. Relabeling is
fully supported in Provenance Replay.

Basic graph operations could also be harnessed to produce algebraic approaches
to provenance replay, in which a user could request, for example, a replay of
all nodes downstream of a feature table, or all nodes upstream of some UUID.
Conceived by Evan Bolyen, this feature could bring significant value to users by
alleviating common issues that arise when updating output executables for
replay.

For example, if a public resource (e.g. a pre-trained classifier or reference
database) that was produced by a third party was used in a QIIME 2 analysis, the
user that is documenting for replay may not have access to the raw data used to
create that resource. Provenance replay will currently document the commands all
the way back to the import of that (potentially unavailable) raw data. A cleaner
solution would be for the user to provide the resource \texttt{.qza} as part of the
reproducibility supplement, and for Provenance Replay to replace the Actions
that created the resource with an import of the resource itself. By allowing
users to exclude parts of their \texttt{DiGraph} from replay, the documentation process
could be made to better support these real-world needs.

Finally, NetworkX provides a rich library of graph operations and algorithms,
which users can apply directly to the \texttt{DiGraph} object (or a copy of it) stored in
\texttt{ProvDag.dag}. Commonly-used operations should be wrapped as methods on
\texttt{ProvDAG} when they might affect inconsistency of internal state, as does union.

\section{Provenance Replay in q2view \parencite[Issue 123]{qiime_2_development_team_issues_2016}}
Graphical interfaces like q2view have a lot of potential to improve user
interactions with provenance data and provide unique opportunities to support
advanced applications of provenance replay. User comments during the focus group
sessions indicate there is significant demand for bundling provenance replay
into q2view, as this would allow users to replay interesting results without
leaving the GUI (or even installing QIIME 2). A graphical interface like this is
probably the most compelling interface for implementing the graph-algebraic
operations described in Section \ref{section:graph_ops}, because describing
the “parts” of an analysis a user would like to run would likely be much easier
with tools like click-and-drag to select.

Unfortunately, there are significant technical hurdles here, so a short-term
implementation is unlikely. The primary challenge is that provenance replay
would likely need to be rebuilt in Rust for compilation to WASM. Dedicated usage
drivers that do not rely on the presence of a QIIME 2 distribution would need to
be built, likely with significant supporting software. Finally, q2view itself
would likely need to be extended to accommodate the visualization of provenance
graphs representing multiple QIIME 2 results.

\section{Annotating transitions into and out of QIIME 2 analysis \parencite[Issue 80]{keefe_issues_2021}}
Multiple focus group participants noted that provenance replay would fail to
fully document analyses in which data was exported from QIIME 2 results for
analysis outside of QIIME 2. This is particularly challenging in cases where the
outside work produces an interim result, which is then re-imported into QIIME 2.

It is beyond the scope of Provenance Replay to manage analyses conducted outside
of QIIME 2. Its potential to reduce the surface area of unreproducible
analytical work is still a benefit. Though there is no clear path forward at
this time, extensions to provenance replay and to the Framework could make it
easier to identify places where external work may have occurred.

\section{Catching bugs}
Because replay has a relatively comprehensive view of the history of a given
result, and because data integrity bugs have been rare in QIIME 2, tooling to
notify users that a given Result might be compromised by a data integrity bug
could be a useful addition to both Provenance Replay and q2view. By querying
DAGs for the presence of characteristics that correlate with loss of integrity
(e.g. failing combinations of OS and a particular version of software) could
help both researchers and reviewers catch potential problems before they go to
press.
