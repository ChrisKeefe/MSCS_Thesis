Bioinformatics workflows are often complex, consisting of dozens or hundreds of
processes, with significant variation possible in computer hardware and software
systems, input data, method selection and parameterization during each step.
This complexity creates known challenges for study organization, reporting, and
reproducibility which may prevent study replication.

Here I present Provenance Replay - software for the documentation and enactment
of in silico reproducibility in QIIME 2, a prominent free and open platform for
microbiome science. QIIME 2 packages the full history (i.e. “provenance”) of
every analysis result within the result itself, including software versions,
methods, parameters, and user-provided metadata. Provenance Replay parses this
captured provenance data into directed acyclic graphs and generates
reproducibility documentation including full-analysis citation lists and
executable scripts capable of replicating the Result(s) in question from the
original input data, providing a robust tool for methods reproducibility. These
executables may also be applied directly to similarly structured data, modified,
or extended, supporting results reproducibility and generalization. This
reproducibility documentation can be used in the automation of repeated
analyses, and has potential to reduce record-keeping, training, and
communication burdens in collaborative research contexts.

Demonstrations, surveys, and focus groups were conducted with an alpha version
of the software, targeting feature elicitation and requirements verification.
In survey results, demonstration participants reported high perceived ease of
use (mean PEOU 5.82 of 7) and high perceived usefulness (mean PU 5.96 of 7), and
a net promoter score of +78.95\%. Overall, respondents report a positive general
attitude toward using Provenance Replay, and a high likelihood of recommending
the software to others.
